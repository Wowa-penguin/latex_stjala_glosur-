\section{KHR 1}
\subsection{Yrðingarökfræði (e. Propositional Logic)}
Yrðingarökfræði er í stutu bara sening sem er sönn eða ósönn.\\
Dæmi væri "er Reykjavík höfuðborg Íslands2" sem er sönn yrðing.\\
Það getur líka verið $5 + 5 = 10$ sem er sönn yrðing en $2 + 2 = 5$ sem er ekki sönn yrðing. 
Ef við erum með dæmi eins og $x + 2 = 9$ það er oppinn yrðing nama við vitum ekki eins og $x = 7$ þá er það sönn yrðing.\\
\subsection{Beiting á rökfræði (e. Applications of Propositional Logic)}
Dæmi um beiting á rökfræði með yrðingarökfræði.\\
p: Skilaboð eru skimuð fyrir vírusum.\\
q: Skilaboð voru send frá óþekktu kerfi.\\
Rituð eru eftirfarandi kerfisskilyrði mep p og q.\\
Þetta er annahvort $p \to q$ eða $q \to p$.\\
\\
a) Skilaboð eru skimuð fyrir vírusum í hvert skifti sem þau berast frá óþekktu kerfi.\\
Þar sem segir að skilaboð voru send frá óþekktu kerfi og p segir að skilaboð voru skimuð þá er þetta dæmi $q \to p$\\
\\
b) Skilaboð voru send frá óþekktu kerfi en voru ekki skimuð fyrir vírusum.\\
Þar sem sagt er að skilaboð voru send frá óþekktu kerfi "en" segjir okkur að það sé eins og $q \wedge \lnot p$.\\
\\
c) Það er nauðsilegt að skima skilaboðin fyrir vírusum í hvert sinn sem þau berast frá óþekktu kerfi.\\
Við sjáum það er "nauðsilegt" sem segir okkur að p þarf að vera True.\\
Þannig það segir okkur að $q \to p$.\\
\\
d) Þegar skilaboð eru send ekki frá óþekktu kerfi eru þau ekki skimuð fyrir vírusum.\\
Við sjáum að dæmið segir "ekki" sem segir okkur að við erum með neitun.\\
Þannig við erum með $(\lnot q) \to (\lnot p)$.
\newpage
\subsection{Jafngildar yrðingar (e. Propositional Equivalences)}
Dæmi um sísönnu sem þýðir að eitthvað er alltaf satt.\\
$(p \wedge q) \to p$ Skoðum þetta í sanntöflu.\\
\begin{tabular}{ c|c|c|c }
    p & q & $p \wedge q$ & $(p \wedge q) \to p$
    \\ \hline
    T & T & T & T 
    \\ \hline
    T & F & F & T
    \\ \hline
    F & T & F & T
    \\ \hline
    F & F & F & T
    \\ \hline
\end{tabular}
\\
Við sjáum að yrðingin er sísana þar sem $(p \wedge q) \to p$ skilar alltaf True.\\
\\ 
Skoðum Jafngildi.\\
Við ætlum að byrja á að umskrifa yrðinguna.$(p \wedge q) \to p$.\\
Við notum regluna um leiðingu sem er $a \to b \equiv \lnot a \vee b$
\begin{align*}
    (p \wedge q) \to p &\equiv (\lnot(p \wedge q)) \to p \quad \text{skv. leiðingar reglu í töflu 7 í bók.}\\
    &\equiv (\lnot p \vee \lnot q) \vee p \quad \text{skv. De Morgan}\\
    &\equiv (\lnot q \vee \lnot p) \vee p \quad \text{skv. Vixlregla}\\
    &\equiv \lnot q \vee (\lnot p \vee p) \quad \text{skv. tengireglu}\\
    &\equiv \lnot q \vee (p \vee \lnot p) \quad \text{skv. Vixlregla}\\
    &\equiv \lnot q \vee T \quad \text{skv. Neitunarreglu}\\
    &\equiv T \quad \text{skv. dominetion law}
\end{align*}
Svona erum við búin að sanna að yrðingin er sísana.\\
Gott að hafa í huga að nota De Morgan law snema en ekki þannig alltaf og reina að hópa saman eins stökum eins og p p. Muna líka að taka einna reglu í einnu og að þær líta eins út og í töfluni.\\
\newpage
\subsection{Umsagnir og kvantarar (e. Predicates and Quantifiers)}
Látum $P(x)$: "x er mera en 5 klukustundir í skólanum" þetta er yrðingafall þar sem formengið er x er allir nemendur í skólanum.\\
Við ætlum að rita eftirfarandi yrðingar sem seningar.\\
\indent a) $\exists x P(x)$: Til er nemandi sem er meyra en 5 klukustundir í skólanum.\\
\indent b) $\forall x P(x)$: Allir nemendur eru meira en 5 klukustundir í skólanum.\\
\indent c) $\exists x \lnot x P(x)$: Til er nemandi sem er ekki meyra en 5 klukustundir í skólanum.\\
\indent \indent  Ef við værum með $\lnot \exists x P(x)$: þar segir okkur að það sé eingin nemandi eða \indent \indent eins og $\forall x$\\
\indent d) $\forall x \lnot P(x)$: Allir nemendur eru ekki meyra en 5 klukustundir í skólanum.\\
Gott er að hafa í huga að $\forall x \equiv \lnot \exists x$ og öfugt $\exists x \equiv \lnot \forall x$\\ 
\\
Annað dæmi.\\
Látum $P(x)$: "x kann rúslensku" og $Q(x)$: "x kann c++".\\
Formengið er x er mengi nemanda í skólanum.\\
Þýðum eftirfarandi seningar yfir í kvantaratrðingar.\\
\indent a) Til er nemandi sem kann bæði rúslensku og c++.\\
\indent\indent Svar: $\exists x (P(x) \wedge Q(x))$. Best er að muna að eftir kvantara þá kemur svigi \indent \indent utanum það sem kvantararinn á við.\\ 
\indent \indent Ef við værum með $\exists x P(x) \wedge Q(x)$ þá erum við að segja að $\exists x$ gildi bara fyrir \indent \indent $P(x)$ en ekki bæði.\\
\indent b) Til er nemandi sem kann rúslensku en ekki c++.\\
\indent \indent Svar: $\exists x (P(x) \wedge \lnot Q(x))$\\
\indent c) Allir nemendur kunna annahvort rúslensku eða c++.\\
\indent \indent Svar: $\forall x (P(x) \vee Q(x))$\\
\indent d) Um gildir að eingin nemandi kann rúslensku eða c++.\\
\indent \indent Svar: $\forall x (\lnot(P(x) \vee Q(x)))$\\
\newpage

\subsection{Margfaldir kvantarar (e. Nested Quantifiers)}
Ritum eftirfarandi yrðingu á Íslensku.\\
Atugum að x og y eru rauntölur.\\
\indent a) $\forall x \exists y (x < y)$: Fyrir hvert x er til y sem er særa en x.\\
\indent b) $\forall x \forall y (((x \geq 0) \wedge (y \geq 0)) \to (xy \geq 0))$: Margfeldi tveggja jákvæða \indent rauntalna eða 0 er jákvæð rauntala eða 0.\\
Atugum að x, y og z eru núna rauntölur menginu.\\
\indent c) $\forall x \forall y \exists z (xy = z)$: Margfaldi tveggja rauntalna er rauntala.\\
\\
Annað Dæmi um margfaldi.\\
Látum $W(x,y)$: "x hefur skoðað y" þar sem formengið x er mengi nemanda og formengið y er vefsíða.\\
Ritum eftirfarandi yrðingar á Íslensku.\\
\indent a) $W(Sara,ruv.is)$: Sara hefur skoðað ruv.is\\
\indent b) $\exists x W(x, stae.is)$: Til er nemandi sem hefur skoðað stae.is.\\
\indent c) $\exists x W(\text{Jóhanes}, x)$: Jóhanes hefur skoðað einhverja vefsíðu.\\
\indent d) $\exists x (W(\text{Assa, x}) \wedge W(\text{Signý, x}))$: Til er vefsíða sem bæði Assa og Signý hafa \indent skðað.\\
\indent e) $\exists x \forall y((x \not = \text{Davíð}) \wedge (W(\text{Davíð, y}) \to W(x,y)))$: það er annar nemandi en \indent Davíð sem hefur skoðað allar vefsíður sem Davíð hefur skoðað.\\
\indent \indent Það þarf að muna ef við erum með $\exists x \forall y$ þá er eitt x sem breytist ekki en ef \indent \indent við erum með $\forall y \exists x$ þá erum við að breyta x eða með marga nemandur.\\
\indent f) $\exists x \exists y \forall z ((x \not = y) \wedge(W(x,z) \leftrightarrow W(y,z)))$: Það eru til tver mismundandi \indent nemandur sem hafa skoða nákvæmlega sömmu síður.\\
\indent \indent Við sjáum að í W er x og y fyrsta stakið sem segir okkur að x og y eru \indent \indent nemendur og z er vefsíður en x og y er ekki sama meneskjan.\\
\newpage
\setcounter{subsection}{6}
\subsection{Sönnunaraðferðir  (e. Introduction to Proofs)}
Dæmi um sönnun að suma tveggja oddatalna er slétt tala.\\
Skilgerining á sléttri tölu er $m = 2k$ og á oddatölu er það $m = 2k+1$\\
Látum m og n vera oddatölur.\\
Þá má rita að $m = 2k+1$ og $n = 2l+1$ þar sem $k,l \in z$\\
Fáum: $m+n = (2k+1) + (2l+1) = 2k + 2l + 2 = 2(k+l+1)$\\
athgum að $k+l+1$ er heil tala, þar með er $2(k+l+1)$ slétt tala og þar með er $m+n$ það líka.\\
Við höfum sannað að summa tveggja oddatalna er slétt tala.\\
\\
Annað dæmi um sönnun.\\
Sönnum að ef n er heil tala og $n^3 + 5$ er oddatala þá er n slétt tala.\\
Við erum með $p \to q$ þar sem p er oddatala ($n^3 + 5$) og q er (n) slétt tala.\\ 
Þar sem $p \to q \equiv \lnot q \to \lnot p$ þá viljum við byrja á að sanna $\lnot p \to \lnot q$.\\
Notum mótskilyrðingu og sýnum fyrst að ef n er oddatala þá er $n^3 + 5$ slétt tala.\\
Látum þá $n = 2k+1$ vera oddatölu $(k \in \mathbb{Z})$.\\
Skoðum: 
\begin{align} 
    n^3 + 5 &= (2k+1)^3 + 5 \\
    (2k+1)(4k^2 + 4k + 1) + 5 &= (8k^3 + 12k^2 + 6k + 1) + 5 \\ 
    8k^3 + 12k^2 + 6k + 6 &= 2(4k^3 + 3k^2 + 3k + 3) 
\end{align}
Nú vitum við að $n^3+5 = 2(4k^3 + 3k^2 + 3k + 3)$ og er heil tala og þar með er $n^3+5 = 2(4k^3 + 3k^2 + 3k + 3)$ slétt tala.\\
Þá höfum við sýnt að ef n er oddatala þá er $n^3+5$ slétt tala. Með mótskilyrðingu fæst þá að ef $n^3+5$ er oddatala þá er n slétt tala.
\newpage