\section*{Dæmi úr síni prófum}
\subsection*{Dæmi 1}
Síni próf 1
\begin{center}
    \vspace{-1cm}
    \begin{tabular}{ c|c|c|c|c|c|c }
        p & q & $\lnot q$ & $p \wedge (\lnot q)$ & $q \to p$ & $(p \wedge (\lnot q)) \leftrightarrow (q \to p)$
        \\ \hline
        T & T & F & F & T & F
        \\ \hline
        T & F & T & T & F & F
        \\ \hline
        F & T & F & F & T & F
        \\ \hline
        F & F & T & F & T & F
    \end{tabular}\\
    \vspace{0.5mm}
    \hspace{-8.7cm}Er ekki sísana
\end{center} \vspace{0.5cm}
Síni próf 2
\begin{center}
    \vspace{-1cm}
    \begin{tabular}{ c|c|c|c|c|c|c }
        p & q & $p \wedge q$ & $p \leftrightarrow q$ & $\lnot (p \wedge q)$ & $(p \leftrightarrow q) \to (\lnot(p \wedge q))$
        \\ \hline
        T & T & T & T & F & F
        \\ \hline
        T & F & F & F & F & T
        \\ \hline
        F & T & F & F & F & T
        \\ \hline
        F & F & F & T & T & T 
    \end{tabular}\\
    \vspace{0.5mm}
    \hspace{-8.7cm}Er ekki sísana
\end{center}

\subsection*{Dæmi 2}
Síni próf 1\\
Gefnar eru eftirfarandi opnar yrðingar\\
$L(x,y)$: nemandi $x$ hefur lokið námskeiðinu $y$.\\
$S(x,y)$: nemandi $x$ má skrá sig í námskeiðið $y$.\\
$U(x)$: nemandi $x$ er útskrifaður.\\
Ritið eftirfarandi staðhæfingar í a) og b) með því að nota eftir því sem við á tilvistarkvantara, allsherjarkvantara, rökfræðileg tákn, og $L(x,y)$, $S(x,y)$ og $U(x)$.\\
a) Einar hefur ekki lokið Tölvuhögun og má ekki skrá sig í Tölvusamskipti.\\
\indent \indent $(\lnot (L(\text{Einar,Tölvuhögun}) \wedge S(\text{x,Tölvusamskipti})))$ \\
b) Enginn nemandi getur útskrifast án þess að hafa lokið bæði Forritun og Reikniritum.\\
\indent \indent $\forall x \lnot U(x) \to L(x, Forritun) \wedge L(x, Reikniritum)$ \textbf{Líklega í réttu átt}\\
c) Gerum ráð fyrir að $\forall x \exists y(L(x,y))$ og $\exists y \forall x (\lnot L(x,y))$ gildi. Úrskurðið þá fyrir hvert af eftirfarandi, hvort það hlýtur að gilda, getur ekki gilt eða óvíst er hvort það gildir, því ekki liggja fyrir nægar upplýsingar til að segja til um það. Setjið einn kross í hverja línu.\\
\indent \indent $\exists x \forall y(\lnot L(x,y))$ Hlýtur að gilda\\
\indent \indent $\forall x \exists y(\lnot L(x,y))$ Hlýtur að gilda\\    
\indent \indent $\exists y \forall x(L(x,y))$ Hlýtur að gilda
\newpage