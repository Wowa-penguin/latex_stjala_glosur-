\section*{Alskonar Glósur frá mér}
\subsection*{Truth table}
\begin{center}
    \begin{tabular}{ c|c|c|c|c|c|c|c|c }
        p & q & $\lnot p$ & $\lnot q$ & $p \wedge q$ & $p \vee q$ & $p \oplus q$ & $p \to q$ & $p \leftrightarrow q$
        \\ \hline
        T & T & F & F & T & T & F & T & T
        \\ \hline
        T & F & F & T & F & T & T & F & F
        \\ \hline
        F & T & T & F & F & T & T & T & F
        \\ \hline
        F & F & T & T & F & F & F & T & T
    \end{tabular}
\end{center}
$p \wedge q$ = AND / $p \vee q$ = OR / $p \oplus q$ = XOR / $p \to q$ = if-then
\subsection*{Kvantarar}
Það eru tver kvanterar sem eru notaðir: $\forall$ og $\exists$.\\
$\forall$: Þessi sentur fyrir allt þannig dæmi væri.\\
\indent $L(x,y)$ Það er nemandi x sem hefur klára námskeð y.\\
\indent $\forall x (L(x,y))$ Segir okkur að allir nemendur hafa klárað námskeð y.\\
$\exists$: Þessi sentur fyrir eitt stak dæmi væri.\\
\indent $L(x,y)$ Það er nemandi x sem hefur klára námskeð y.\\
\indent $\exists x (L(x,y))$ Segir okkur að það er til nemandi sem hefur klára námskeð y.
\subsection*{Mengi}
Gott að hafa í huga að eftirfarandi er satt.\\
$A \vee B \equiv A \cup B$ \quad $A \wedge B \equiv A \cap B$\\

Þýðir að n er jákvæð heiltala $n \in \mathbb{Z}_+$ 

Tvíund\\

\begin{align*}
    \displaystyle \sum_{i = 2}^{2} \sum_{j=0}^{2} \lceil \log_2(i+3j)\rceil &= \\
    \displaystyle \sum_{i=1}^{2} i \cdot \sum_{j=0}^{2} \lceil \log_2(3j)\rceil &= \\
    \displaystyle \sum_{i=1}^{2}(i(\lceil \log_2(3\cdot1)\rceil + \lceil \log_2(3\cdot2)\rceil) &= \\
\end{align*}

Spehilvirk 
samhevrf
andsamhverf
gegnvirk
\newpage