\section{KHR}
\vspace*{-0.4em}
\subsection{Inngangur að strjálum líkindum (e. An Introduction to Discrete Probability) }
\subsubsection{Líkindafræði}
\vspace*{-0.4em}
Hvað eru líkurnar á að daga ás?\vspace*{0.5em}\\
Látum A vera atburðinn að draga ás.\\
Látum S vera úrtaksrúmið að draga spil.\\
Það er mikilvægt í svona dæmum að búa til skilgreiningu á atburðum.\vspace*{0.5em}\\
Það eru 4 leiðir til að draga ás og 52 spil til að draga þanni að.\\
Þá er |A| = 4 \quad |S| = 52\vspace*{0.5em}

\hbox{Þá fæst: $p(A) = \frac{ |A| }{ |S| }$ (stærð á því sem við viljum að gerist deilt í stærð sem getur garst)}\vspace*{0.5em}

$p(A) = \frac{ |A| }{ |S| } = \frac{4}{52} = \frac{1}{13}\approx 7.7\%$
\vspace*{-0.4em}
\subsubsection{Líkindafræði, póker}
\vspace*{-0.4em}
Hverjar eru líkurnar að póker hönd innihaldi tígultvist, spaðaþrist, hjartasexu, lauftíu og hjartakóng?\vspace*{0.5em}\\
Látum A vera atburðinn að þessi hönd sé gefin.\\
Látum S vera úrtaksrúmið.\\
\hbox{|A| = 1 (Því það er bara einn svona hönd) \quad |S| = C(52, 5) (þar sem við viljum 5 spil)}\vspace*{0.5em}

Við vitum hvernig á að reikna $((52, 5) = \frac{52!}{(52-5)! \cdot 5!} = 2.518.060$
\begin{equation*}
    \text{Fáum: } p(A) = \frac{ |A| }{ |S| } = \frac{1}{C(52,5)} = \frac{1}{2.518.060} \approx 0,000038\%
\end{equation*}\vspace*{1em}\\
Annað dæmi.\qquad Formúla: $p(A) = 1- p(\overline{A}) $\vspace*{0.3em}\\
Hverjar eru líkurnar á að fimm spila pókerhendi innihaldi ekki hjartadrottningu?\vspace*{0.3em}\\
Látum A vera atburðinn að höndinn innihaldi ekki hjartadrottningu.\\
Látum S vera úrtaksrúmið að vera gefinn 5 spila hönd.\\
Atugum að |S| = C(52, 5). Við tökum svo hjartadrottningu úr stokknum og sjáum að |A| = C(51, 5).
\begin{equation*}
    p(A) = \frac{ |A| }{ |S| } = \frac{C(51,5)}{C(52,5)} = \frac{51!}{46! \cdot 5!} / \frac{52!}{47! \cdot 5!} = \frac{51! \cdot 47! \cdot 5!}{52! \cdot 46! \cdot 5!} = \frac{47}{52} \approx 90.4\%
\end{equation*}\\
\newpage
\hspace*{-1.3em}Annað dæmi.\qquad Formúla: $p(A) = 1- p(\overline{A}) $\vspace*{0.3em}\\
Hverjar eru líkurnar á því að pókerhönd innihaldi amk einn ás?\vspace*{0.3em}\\
Algeng vila sem gerist er að tvítelja svona: $\underline{1\heartsuit} \ldots \underline{1\spadesuit } \ldots$\\
Það sem gerist er að við töldum tvo ása en við vildum bara telja einn.\vspace*{0.5em}\\
\hbox{Látum A vera atburðinn amk einn ás sé á höndinni. Þá er $\overline{A}$ að einginn ás sé á hönd.}
Úrtaksrúmið er S sem hefur stærðina |S| = C(52, 5).\vspace*{0.4em}

$|\overline{A}| = C(48,5)$ Þar sem við tökum alla ása úr stokknum.
\begin{equation*}
    p(A)= 1- p(\overline{A}) = 1 - \frac{ |A| }{ |S| } = 1 - \frac{C(48,5)}{C(52,5)} \approx 34\%
\end{equation*}
Yfirlegt er þetta nógu gott svar nema annað er tekið fram.\\



\newpage
\setcounter{section}{8}
\setcounter{subsection}{0}
\subsection{Beiting rakningarformúla  (e. Applications of Recurrence Relations) upp að (e. up to)}
Skilaboð eru send sem röð merkja sem eri annað hvort $1_{ms}$ eða $2_{ms}$ af lengd.\vspace*{0.5em}\\
Finnum fjölda Skilaboð sem eru $n_{ms}$.\\
Látum $a_n$ tákna fjölda mögulega Skilaboða sem eru $n_{ms}$ af lengd.\vspace*{0.5em}

a) Finnum rakningarformúlu:
\hspace*{-1.3em}Skoðum skilaboð af lengf n.\\
\begin{multicols}{2}
    \hspace*{-1.3em}Ef Skilaboð byrja á merki af lengd 1 þá eru $n - 1 ms$ eftir.\\
    Þær má filla á $a_{n-1}$ vegu.\\
    Sama fæst að ef skilaboð byrja á merki af lengd 2 má klára á $a_{n-2}$ vegu.\\
    Þá fæst í heildina eru $a_n = a_{n-2}+ a_{n-1}$ möguleg skilaboð af lengd n þar sem $n \geq 2$
    \columnbreak

    \hspace*{3em}\begin{tabular}{ |c|c| }
        \hline
        1 & \qquad n-1 \hspace*{3em}
        \\\hline
    \end{tabular}\vspace*{1em}\\ 
    \hspace*{4.3em}\begin{tabular}{ |c|c| }
        \hline
        \quad2  \quad & \qquad n-2 \hspace*{3em}
        \\\hline
    \end{tabular}
    \begin{align*}
        &a_n = a_{n-2}+ a_{n-1}\\
        &a_0=1 \quad a_1=1\\
        &n \geq 2
    \end{align*}
\end{multicols}
b) Finnum upphafsgildi, atugum að það eru aðeins ein tóm skilaboð tómu \hspace*{2.7em}skilaboðin svo $a_0=1$.

Það eru einnig þarf ein skilaboð af lengd 1 skilaboðin 1 svo $a_1 = 1$\\
\hspace*{1.3em}Upphafsskilirðin eru $a_0=1$ og $a_1=1$\vspace*{0.5em}

c) Finnum fjölda skilaboða af lengd $10_{ms}(a_{10})$
\begin{align*}
    &a_0 = 1\\
    &a_1 = 1\\
    &a_2 = a_{2-2}+ a_{2-1} = a_0 + a_1 = 1 + 1 = 2\\
    &a_3 = a_{3-2}+ a_{3-1} = a_0 + a_1 = 1 + 2 = 3 \\
    &a_4 = 2 + 3 = 5\\
    &a_5 = 3 + 5 = 8\\
    &a_6 = 5 + 8 = 13\\
    &\vdots\\
    &a_{10} = 34 + 55 = 89
\end{align*}
Það eru því 89 möguleg skilaboð af lengd $10_{ms}$

\newpage
\subsubsection{Annað dæmi um rakningarformúlur}
Hversu margar mögulegar hellur má leggja með rauðum, bláum og grænum hellum ef hellurnar eru lagðar í einna röð og það mega ekki vera tvær rauðar hlið við hlið.\vspace*{1em}

a) Finnum rakningarformúlu: Látum $a_n$ tákna fjölda löglegra lagna af lengd n.
\begin{multicols}{2}
    \hspace*{-1.3em}Skoðum hellulögn af lengd n.\\
    Ef fyrsta hellan er annahvort græn eða blá má setja hvaða löglegu lögn sem er af lengd n-1 a fyrir aftan, það er $a_{n-1}$ mögulegar fyrir hvorn lit.\\
    Ef lögnin byrjar á rauðri hellu verður næsta hella að vera græn eða blá. Síðan má klára með löglegri lögn að lengd n-2, þar er $a_{n-2}$ fyrir hvorn lit á eftir rauðari hellu.\\
    Í heldina eru þá $a_n = 2(a_{n-2}+ a_{n-1})$ margar löglegra lagnir á lengd n.

    \columnbreak

    \begin{center}
        n hellur
    \end{center}
    \hspace*{4.3em}\begin{tabular}{ c|c|c|c| }
        & & 2x
        \\\hline
        $2a_{n-2}$ & R & G/B & \hspace*{1em}$a_{n-2}$ \hspace*{1em}
        \\\hline
    \end{tabular}\vspace*{1em}\\ 
    \hspace*{4.3em}\begin{tabular}{ c|c|c| }
        \hline
        $2a_{n-1}$ & G & \hspace*{1em}$a_{n-1}$ \hspace*{1em}
        \\\hline
    \end{tabular}\vspace*{1em}\\ 
    \hspace*{4.3em}\begin{tabular}{ c|c|c| }
        \hline
        $2a_{n-1}$ & B & \hspace*{1em}$a_{n-1}$ \hspace*{1em}
        \\\hline
    \end{tabular}
\end{multicols}

b) Finnum upphafsskilirðin: 

Atugum að tóma hellulögnin er lögleg svo $a_0 = 1$

Atugum líka að allar legnir með einni hellu eru líka löglegar svo $a_1 =3$\vspace*{0.5em}


c) Hvað eru margar löglegar lagnir af lengd 7? 

Formúlan er þá: $n \geq 2$
\begin{align*}
    a_n &= 2(a_{n-2}+ a_{n-1})\\
    a_0 &= 1 \\ 
    a_1 &= 3 \\ 
    a_2 &= 2(a_{n-2}+ a_{n-1}) = 2(1+3) = 8\\
    a_3 &= 2(2+8) = 22\\
    a_4 &= 2(8+22) = 60\\
    a_5 &= 2(22+60) = 164\\
    a_6 &= 2(60+164) = 448\\
    a_7 &= 2(164+448) = 1224
\end{align*}
Þannig það eru 1224 löglegra hellulagnir af lengd 7.
\newpage