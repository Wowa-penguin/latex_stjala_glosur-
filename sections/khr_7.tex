\section{KHR}
\vspace*{-0.4em}
\subsection{Inngangur að strjálum líkindum (e. An Introduction to Discrete Probability) }
\subsubsection{Líkindafræði}
\vspace*{-0.4em}
Hvað eru líkurnar á að daga ás?\vspace*{0.5em}\\
Látum A vera atburðinn að draga ás.\\
Látum S vera úrtaksrúmið að draga spil.\\
Það er mikilvægt í svona dæmum að búa til skilgreiningu á atburðum.\vspace*{0.5em}\\
Það eru 4 leiðir til að draga ás og 52 spil til að draga þanni að.\\
Þá er |A| = 4 \quad |S| = 52\vspace*{0.5em}

\hbox{Þá fæst: $p(A) = \frac{ |A| }{ |S| }$ (stærð á því sem við viljum að gerist deilt í stærð sem getur garst)}\vspace*{0.5em}

$p(A) = \frac{ |A| }{ |S| } = \frac{4}{52} = \frac{1}{13}\approx 7.7\%$
\vspace*{-0.4em}
\subsubsection{Líkindafræði, póker}
\vspace*{-0.4em}
Hverjar eru líkurnar að póker hönd innihaldi tígultvist, spaðaþrist, hjartasexu, lauftíu og hjartakóng?\vspace*{0.5em}\\
Látum A vera atburðinn að þessi hönd sé gefin.\\
Látum S vera úrtaksrúmið.\\
\hbox{|A| = 1 (Því það er bara einn svona hönd) \quad |S| = C(52, 5) (þar sem við viljum 5 spil)}\vspace*{0.5em}

Við vitum hvernig á að reikna $((52, 5) = \frac{52!}{(52-5)! \cdot 5!} = 2.518.060$
\begin{equation*}
    \text{Fáum: } p(A) = \frac{ |A| }{ |S| } = \frac{1}{C(52,5)} = \frac{1}{2.518.060} \approx 0,000038\%
\end{equation*}\vspace*{1em}\\
Annað dæmi.\qquad Formúla: $p(A) = 1- p(\overline{A}) $\vspace*{0.3em}\\
Hverjar eru líkurnar á að fimm spila pókerhendi innihaldi ekki hjartadrottningu?\vspace*{0.3em}\\
Látum A vera atburðinn að höndinn innihaldi ekki hjartadrottningu.\\
Látum S vera úrtaksrúmið að vera gefinn 5 spila hönd.\\
Atugum að |S| = C(52, 5). Við tökum svo hjartadrottningu úr stokknum og sjáum að |A| = C(51, 5).
\begin{equation*}
    p(A) = \frac{ |A| }{ |S| } = \frac{C(51,5)}{C(52,5)} = \frac{51!}{46! \cdot 5!} / \frac{52!}{47! \cdot 5!} = \frac{51! \cdot 47! \cdot 5!}{52! \cdot 46! \cdot 5!} = \frac{47}{52} \approx 90.4\%
\end{equation*}\\
\newpage
\hspace*{-1.3em}Annað dæmi.\qquad Formúla: $p(A) = 1- p(\overline{A}) $\vspace*{0.3em}\\
Hverjar eru líkurnar á því að pókerhönd innihaldi amk einn ás?\vspace*{0.3em}\\
Algeng vila sem gerist er að tvítelja svona: $\underline{1\heartsuit} \ldots \underline{1\spadesuit } \ldots$\\
Það sem gerist er að við töldum tvo ása en við vildum bara telja einn.\vspace*{0.5em}\\
\hbox{Látum A vera atburðinn amk einn ás sé á höndinni. Þá er $\overline{A}$ að einginn ás sé á hönd.}
Úrtaksrúmið er S sem hefur stærðina |S| = C(52, 5).\vspace*{0.4em}

$|\overline{A}| = C(48,5)$ Þar sem við tökum alla ása úr stokknum.
\begin{equation*}
    p(A)= 1- p(\overline{A}) = 1 - \frac{ |A| }{ |S| } = 1 - \frac{C(48,5)}{C(52,5)} \approx 34\%
\end{equation*}
Yfirlegt er þetta nógu gott svar nema annað er tekið fram.\\



 
\newpage
\setcounter{section}{8}
\setcounter{subsection}{0}
\subsection{Beiting rakningarformúla  (e. Applications of Recurrence Relations) upp að (e. up to)}

\newpage