\section{KHR}

\subsection{Heiltöludeiling og mátreikningur (e. Divisibility and Modular Arithmetic)}

\subsubsection{Heiltöludeiling}
Finnum kvóta og afgang þegar. \quad Regla: $x = q \cdot d + r$\\
a) 19 er deilt með 7: $19 = 2 \cdot 7 + 5$ Þar sem kvótinn er 2 og afgangurinn er 5.\\
b) -111 er deilt með 11: $-111 = -11 \cdot 11 + 10$ Þar sem kvótinn er -11 og afgangurinn er 10. Afgangurinn verður alltaf að vera stæri en 0.\vspace*{1em}\\
Skoðum aðra leið.\\
e) 0 er deilt með 19: $q = 0/19=0 r = q \cdot d - x = 0 \cdot 19 - 0 = 0$\\
Athugum að $0=0\cdot19+0$ þannig að kvótinn og afgangurinn er 0.\vspace*{1em}\\
Aukadæmi.\\
Þegar 36 er deilt með 5.\\
$q = \left\lfloor 36/5 \right\rfloor = \left\lfloor 7,2 \right\rfloor = 7$\\
$r = x -q \cdot d = 36- 7 \cdot 5 = 36 - 35 = 1$\\
Athugum að $36=7\cdot5+1$\\
þá er kvótinn 7 og afgangurinn 1
\subsubsection{Reiknirit Evklíðs og mod}
Finnum eftirfarandi gildi:\\
a) 13 mod 3 = 1 því að $13 = 4 \cdot 3 +1$\\
b) -97 mod 11 = 2\\
c) 155 mod 19 = 3

\newpage
\setcounter{subsection}{2}
\subsection{Prímtölur og stærsti samdeilirinn (e. Primes and Greatest Common Divisor)}
Allar sléttar tölur nema 2 eru ekki prímtölur.\\
Ákvöðum hvort eftirfarandi tölur eru prímtölur. Tölur sem eru bara deilanlegar með 1 og sjálfan sér.\\
a) Talan 21 er ekki prímtala þar sem $21 = 3 \cdot 7$.\\
b) Talan 97 þar sem $11 > \sqrt{97} = 9,8$ segir okkur að við þrufum bara að skoða þær prímtölur sem fara upp að 11 en ekki 11 sjálf. Þannig talan er prímtala þar sem hún er ekki deilanleg með 2,3,5 eða 7 sem eru allar prímtölur upp að 11.\\
c) Talan 143 byrjum á að skoða $11 > \sqrt{143} = 11.99$ svo við þurfum að skoða 11 svo þá getum við séð að $11 \cdot 13 = 143$ þannig 143 er ekki prímtala.

\subsubsection{Frumþáttun}
Finnum frumþáttun eftirfarandi talna. Frumþáttun er að deila tölu en oft og hægt er.\\
a) Tala 88 
\vspace*{-1em}\begin{align*}
    88 &= 2 \cdot 44 \\
    &= 2 \cdot 2 \cdot 22 \\
    &= 2 \cdot 2 \cdot 2 \cdot 11 \\
    &= 2^3 \cdot 11 
\end{align*}
Þannig að frumþáttun 88 er $88 = 2^3 \cdot 11$\\
b)Talan 1001 
\vspace*{-1em}\begin{align*}
    1001 &= 7 \cdot 143 \\
    &= 7 \cdot 11 \cdot 13
\end{align*}
Þannig að Frumþáttun 1001 er $1001 = 7 \cdot 11 \cdot 13$
\subsubsection{Ósamþátta}
Finnum af eftirfarandi tölur eru ósamþátta\\
a) 11, 15 og 19: Við bryjum að finna Frumþáttun talnana þar sem 11 = 11 þar sem það er prim tala og sama með 19 en $15 = 3 \cdot 5$\\
Þar sem engin tala deilir Frumþætti með hvor annarri þá eru því ósamþátta tvær og tvær.\\
b) 14 15 og 21: $14 = 2 \cdot 7$, $ 15=3\cdot 5 $ og $21 = 3\cdot 7$\\
Þannig að 14 og 21 eru samþátta og 15 og 21 eru samþátta. Vegna 7 og 3.\\
\newpage
\subsubsection{Stærsti samdeilir, gcd}
Finnum stæðsta samdeiginlega deilir eftirfarandi talan.\\
a) Tölurnar $3^7 \cdot 5^3 \cdot 7^3$ og $2^{11} \cdot 3^5 \cdot 5^9$\\
Báðar tölurnar má deila með $3^5$ og $5^3$ Þannig að stæðsti sameiginlega deilir er $3^5 \cdot 5^3$\\
b) Tölurnar $11 \cdot 13 \cdot 17$ og $2^9 \cdot 3^7 \cdot 5^5 \cdot 7^3$\\
Tölurnar eru ósamþátta og því er stæðsti deilir þerra 1.
\subsubsection{Evklíðs}
Notum reiknirit evklíðs til að finna stæðsta sameiginlega deilir.\\
a) Finnum gcd(111,201) fáum:
\vspace*{-1em}\begin{align*}
    201 &= 1 \cdot 111 + 90 \\
    111 &= 1 \cdot 90 + 21 \\
    90 &= 4 \cdot 21 + 6 \\
    21 &= 3 \cdot 6 + 3 \\
    6 &= 2 \cdot 3 + 0 \\
\end{align*}Þannig að við sjáum að skv reiknirit evklíðs er gcd(111, 201) = 3\\
b) Finnum gcd(1001, 1331) fáum:
\vspace*{-1em}\begin{align*}
    1331 &= 1 \cdot 1001 + 330 \\
    1001 &= 3 \cdot 330 + 11 \\
    330 &= 30 \cdot 11 + 0 
\end{align*}
Þannig að við sjáum að skv reiknirit evklíðs er gcd(1001, 1331) = 11\\

\newpage
\setcounter{subsection}{4}
\subsection{Notkun leifajafna (e. Applications of Congruences)}
Bílastæði með 31 stæði (0-30). \\Finnið stæði fyrir gesti með tætifalli $h(k) = k \mod 31$ út frá bílnúmeri 317, 918, 007, 100, 111 og 310.\\
a) Fráum:
\begin{align*}
    317 = 31 \cdot 10 + 7 \text{ þannig að } 317 &\equiv 7 \mod 31\\
    918 = 31 \cdot 29 + 19 \text{ þannig að } 918 &\equiv 19 \mod 31\\
    7 = \text{ } \ldots \text{ } 7 &\equiv 7 \mod 31\\
    100 = \text{ } \ldots \text{ } 100 &\equiv 7 \mod 31\\
    111 = \text{ } \ldots \text{ } 111 &\equiv 18 \mod 31\\
    310 = \text{ } \ldots \text{ } 310 &\equiv 0 \mod 31\\
\end{align*} Gestirnir fá úthlutað stæðum talan fyrir fram mod 7 19 7 7 18 0.\\
b) Hvað ef stæðið er upptekið.\\
Þá skoðar hann næsta í hring (mod 31)\\
Hann finnur $h(k+a)$ þar sem a er fjöldi stæða sem henn hefur prófað.\\
Hættum þegar a er = 31.
\subsubsection{Gervislempirunnuna}
Finnum gervislembirunnuna sem frameiðslufallið $x_{n+1} = 3 \cdot x_n mod 11$ ef upphafsgildið er $x_0 = 2$\\
$x_0 = 2$\\
$x_1 = 3 \cdot x_0 \mod 11 = 3 \cdot 2 \mod 11 = 6 \mod 11 = 6$\\
$x_2 : 3 \cdot x_1 = 3 \cdot 6 = 18 \equiv 7 \mod 11 = 7$ Þar sem $7 = 18 \mod 11$\\
$x_3 : 3 \cdot x_2 = 3 \cdot 7 = 21 \equiv 10 \mod 11 = 10$ þar sem $10 = 21 \mod 11$ \\
$x_4 : 3 \cdot x_3 = 3 \cdot 10 = 30 \equiv 8 \mod 11 = 8$ þar sem $8 = 30 \mod 11$.\\
$x_5 : 3 \cdot x_4 = 3 \cdot 8 = 24 \equiv 2 \mod 11 = 2$ þar sem $2 = 24 \mod 11$\\
Athugm að $x_0 = x_5 = 2$ þannig að við erum komin með hring gervislembirunnuna endurtekur sig og er því $2,6,7,10,8,2,6,7,10,8,2 \ldots$\\
\newpage